\documentclass{article}

% set font encoding for PDFLaTeX or XeLaTeX
\usepackage{graphicx}
\usepackage{ifxetex}
\usepackage{float}
\usepackage{}
\ifxetex
  \usepackage{fontspec}
\else
  \usepackage[T1]{fontenc}
  \usepackage[utf8]{inputenc}
  \usepackage{lmodern}
\fi

% used in maketitle
\title{Reporte: Actividad 7}
\author{Luisa Julieta Casas Cervantes}
\date{23 de Marzo del 2018}
% Enable SageTeX to run SageMath code right inside this LaTeX file.
% documentation: http://mirrors.ctan.org/macros/latex/contrib/sagetex/sagetexpackage.pdf
% \usepackage{sagetex}

\begin{document}
\maketitle
\section{Introducción}
Esta es una continuación de la actividad previa (Actividad 6) en la cual trabajamos con sistemas de dos masas acopladas, excepto que en esta ocasión, se añade un factor que cambia el sistema y lo vuelve no lineal.
\section{Python y Gráficas resultantes}
Definición de variables
\begin{figure}[H]
\includegraphics[height=7cm]{def1.png}
\end{figure}
\begin{figure}[H]
\includegraphics[height=10cm]{var31.png}
\end{figure}
Ejemplo 3.1
\begin{figure}[H]
\includegraphics[height=10cm]{311.png}
\end{figure}
\begin{figure}[H]
\includegraphics[height=10cm]{312.png}
\end{figure}
\begin{figure}[H]
\includegraphics[height=10cm]{313.png}
\end{figure}
\begin{figure}[H]
\includegraphics[height=10cm]{314.png}
\end{figure}
\begin{figure}[H]
\includegraphics[height=10cm]{315.png}
\end{figure}
Ejemplo 3.2
\begin{figure}[H]
\includegraphics[height=10cm]{var32.png}
\end{figure}
\begin{figure}[H]
\includegraphics[height=10cm]{321.png}
\end{figure}
\begin{figure}[H]
\includegraphics[height=10cm]{322.png}
\end{figure}
\begin{figure}[H]
\includegraphics[height=10cm]{323.png}
\end{figure}
\begin{figure}[H]
\includegraphics[height=10cm]{324.png}
\end{figure}
Ejemplo 3.3
\begin{figure}[H]
\includegraphics[height=10cm]{var33.png}
\end{figure}
\begin{figure}[H]
\includegraphics[height=10cm]{331.png}
\end{figure}
\begin{figure}[H]
\includegraphics[height=10cm]{332.png}
\end{figure}
\begin{figure}[H]
\includegraphics[height=10cm]{333.png}
\end{figure}
\begin{figure}[H]
\includegraphics[height=10cm]{334.png}
\end{figure}
Ejemplo 4.1
\begin{figure}[H]
\includegraphics[height=10cm]{var41.png}
\end{figure}
\begin{figure}[H]
\includegraphics[height=10cm]{411.png}
\end{figure}
\begin{figure}[H]
\includegraphics[height=10cm]{412.png}
\end{figure}
\begin{figure}[H]
\includegraphics[height=10cm]{413.png}
\end{figure}
\begin{figure}[H]
\includegraphics[height=10cm]{414.png}
\end{figure}
\section{Apéndice}
\begin{enumerate}
\item ¿Qué más te llama la atención de la actividad completa? ¿Que se te hizo menos interesante?
\\Las figuras que salen de movimientos oscilatorios. Lo que no me gustó es que no cambian mucho con diferentes datos.
\item ¿De un sistema de masas acopladas como se trabaja en esta actividad, hubieras pensado que abre toda una nueva área de fenómenos no lineales?
\\Me lo sospechaba.
\item ¿Qué propondrías para mejorar esta actividad? ¿Te ha parecido interesante este reto?
\\Datos más variados para ver las figuras.
\item ¿Quisieras estudiar mas este tipo de fenómenos no lineales?
\\Teóricamente sí.
\end{enumerate}
\end{document}
