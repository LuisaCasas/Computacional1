\documentclass{article}

% set font encoding for PDFLaTeX or XeLaTeX
\usepackage{graphicx}
\usepackage{ifxetex}
\usepackage{float}
\usepackage{}
\ifxetex
  \usepackage{fontspec}
\else
  \usepackage[T1]{fontenc}
  \usepackage[utf8]{inputenc}
  \usepackage{lmodern}
\fi

% used in maketitle
\title{Reporte: Actividad 6}
\author{Luisa Julieta Casas Cervantes}
\date{14 de Marzo 2018}
% Enable SageTeX to run SageMath code right inside this LaTeX file.
% documentation: http://mirrors.ctan.org/macros/latex/contrib/sagetex/sagetexpackage.pdf
% \usepackage{sagetex}

\begin{document}
\maketitle
\section{Introducción}
Con la intención de brindar un enfoque a fenómenos físicos, comenzaremos por estudiar oscilaciones basándonos en un texto provisto por el profesor en clase.
\\El texto nos presenta distintos ejercicios a realizar, los cuales llevaremos a cabo en la plataforma de Jupyter lab basándonos en un artículo que maneja sistemas de masas acopladas de una manera similar, y lo adaptaremos a cada ejercicio.
\section{Código Python}
\begin{figure}[H]
\includegraphics[width=\linewidth]{def.png}
\end{figure}
Para el ejemplo 2.1:
\begin{figure}[H]
\includegraphics[height=10cm]{var21.png}
\end{figure}
Movimiento sincronizado x1 y x2:
\begin{figure}[H]
\includegraphics[width=\linewidth]{sincro21.png}
\end{figure}
Gráfica para x1 y x2:
\begin{figure}[H]
\includegraphics[width=\linewidth]{circ21.png}
\end{figure}
\begin{figure}[H]
\includegraphics[width=\linewidth]{x1221.png}
\end{figure}
Ejemplo 2.2:
\begin{figure}[H]
\includegraphics[height=10cm]{var22.png}
\end{figure}
Movimiento sincronizado x1 y x2:
\begin{figure}[H]
\includegraphics[width=\linewidth]{sincro22.png}
\end{figure}
Gráfica para x1 y x2:
\begin{figure}[H]
\includegraphics[width=\linewidth]{circ22.png}
\end{figure}
\begin{figure}[H]
\includegraphics[width=\linewidth]{x1222.png}
\end{figure}
Ejemplo 2.3:
\begin{figure}[H]
\includegraphics[height=10cm]{var23.png}
\end{figure}
Movimiento sincronizado x1 y x2:
\begin{figure}[H]
\includegraphics[width=\linewidth]{sincro23.png}
\end{figure}
Gráfica x1:
\begin{figure}[H]
\includegraphics[width=\linewidth]{x123.png}
\end{figure}
Gráfica x2:
\begin{figure}[H]
\includegraphics[width=\linewidth]{x223.png}
\end{figure}
Gráfica x1 vs x2:
\begin{figure}[H]
\includegraphics[width=\linewidth]{x1223.png}
\end{figure}
Ejemplo: 2.4
\begin{figure}[H]
\includegraphics[height=10cm]{var24.png}
\end{figure}
Movimiento sincronizado x1 y x2:
\begin{figure}[H]
\includegraphics[width=\linewidth]{sincro24.png}
\end{figure}
\section{Gráficas Resultantes}
Ejemplo 2.1:
\begin{figure}[H]
\includegraphics[height=6cm]{isincro21.png}
\end{figure}
\begin{figure}[H]
\includegraphics[height=6cm]{icirc21.png}
\end{figure}
\begin{figure}[H]
\includegraphics[height=6cm]{ix1221.png}
\end{figure}
Ejemplo 2.2:
\begin{figure}[H]
\includegraphics[height=6cm]{isincro22.png}
\end{figure}
\begin{figure}[H]
\includegraphics[height=6cm]{icirc22.png}
\end{figure}
\begin{figure}[H]
\includegraphics[height=6cm]{ix1222.png}
\end{figure}
Ejemplo 2.3:
\begin{figure}[H]
\includegraphics[height=6cm]{isincro23.png}
\end{figure}
\begin{figure}[H]
\includegraphics[height=6cm]{icirc23.png}
\end{figure}
\begin{figure}[H]
\includegraphics[height=6cm]{icirc223.png}
\end{figure}
\begin{figure}[H]
\includegraphics[height=6cm]{ix1223.png}
\end{figure}
Ejemplo 2.4:
\begin{figure}[H]
\includegraphics[height=6cm]{isincro24.png}
\end{figure}
\section{Apéndice}
\begin{enumerate}
\item ¿En general te pareció interesante esta actividad de modelación matemática? ¿Qué te gustó mas? ¿Qué no te gustó?
\\Me parece muy interesante ya que es darle enfoque a la carrera. Me gustó analizar las diferencias entre el ejemplo y la actividad a realizar. 
\item La cantidad de material te pareció ¿bien?, ¿suficiente?, ¿demasiado?
\\Me parece una cantidad normal de material.
\item ¿Cuál es tu primera impresión de Jupyter Lab? 
\\No lo veo tan diferente al Jupyter Notebook.
\item Respecto al uso de funciones de SciPy, ¿ya habías visto integración numérica en tus cursos anteriores? ¿Cuál es tu experiencia?
\\No lo había visto en FORTRAN, pero sí en otras materias.
\item El tema de sistema de masas acopladas con resortes, ¿ya lo habías resuelto en tu curso de Mecánica 2? 
\\Lo habíamos resuelto. Pero no con ecuaciones diferenciales.
\item ¿Qué le quitarías o agregarías a esta actividad para hacerla más interesante y divertida? 
\\Le agregaría ejemplos personalizados. 
\end{enumerate}
\end{document}
