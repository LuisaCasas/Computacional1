\documentclass{article}

% set font encoding for PDFLaTeX or XeLaTeX
\usepackage{ifxetex}
\usepackage[utf8]{inputenc}
\usepackage{amsmath}
\usepackage{graphicx}
\ifxetex
  \usepackage{fontspec}
\else
  \usepackage[T1]{fontenc}
  \usepackage[utf8]{inputenc}
  \usepackage{lmodern}
\fi

% used in maketitle
\title{La Atmósfera Terrestre}
\author{Luisa Casas}
\date {25 de Enero 2018}
% Enable SageTeX to run SageMath code right inside this LaTeX file.
% documentation: http://mirrors.ctan.org/macros/latex/contrib/sagetex/sagetexpackage.pdf
% \usepackage{sagetex}

\begin{document}
\maketitle
\section{Introducción: ¿Qué es la Atmósfera?}
La atmósfera de la Tierra es la capa de gases (comúnmente conocida como "aire") que rodea la Tierra y es retenida por la gravedad de esta.
La atmósfera terrestre, además de ser en gran parte responsable de la vida en nuestro planeta (creando presión permitiendo que el agua líquida exista en nuestro planeta), cumple con otras funciones, entre las cuales se encuentran la variación de temperatura entre día y noche, la retención de calor que calienta la superficie y la absorción de radiación ultravioleta.
\begin{figure}[h!]
    \includegraphics[width=\linewidth]{introduccion.jpg}
\end{figure}
\subsection{Características de la Atmósfera}
La atmósfera de la Tierra tiene una masa de aproximadamente $5.15*10^{18}$kg, tres cuartas partes de lo cual se encuentra dentro de aproximadamente $11$km de la superficie. La atmósfera se vuelve cada vez más delgada conforme aumenta la altura, sin un límite definido entre la atmósfera y el espacio exterior. Los efectos atmosféricos se vuelven notorios a través de una reentrada atmosférica (movimiento de un objeto desde el espacio exterior a la Tierra) de una nave a una altitud de 120km. Varias capas pueden ser distinguidas en la atmósfera, basadas principalmente en temperatura y composición
\section{Capas Principales}
\begin{figure}[h!]
\includegraphics[width=\linewidth]{capas2.jpg}
\end{figure}
\subsection{Exosfera}
Esta capa está compuesta principalmente de extremadamente bajas densidades de hidrógeno, helio y multitudes de moléculas pesadas tales como nitrógeno, oxígeno y dióxido de carbono más cerca de la exobase(límite inferior de la exosfera).  
Dado que las moléculas en la exosfera se encuentran tan separadas que pueden viajar kilómetros sin colisionar entre ellas, ésta ya no se comporta como un gas y las partículas frecuentemente escapan al espacio. 
La exosfera contiene la mayoría de los satélites orbitando la Tierra.
\subsection{Termósfera}
La termósfera es la segunda capa más alta de la atmósfera terrestre. Su temperatura aumenta gradualmente con la altura. La inversión de temperatura en esta capa ocurre debido a la extremadamente baja densidad de sus moléculas, y aquí la temperatura puede aumentar hasta 1500 grados Celsius.
La termósfera es una capa completamente despejada (sin nubes) y libre de cualquier vapor de agua. Sin embargo, fenómenos no hidro-meteorológico tales como las auroras boreales son vistos ocasionalmente en esta capa. 
\subsection{Mesósfera}
Es la tercera capa más alta de la atmósfera terrestre a una altitud de entre 80 y 85km por encima del nivel del mar. 
En esta capa, las temperaturas decrecen con la altitud, haciéndola el lugar más frío de la Tierra con una temperatura promedio de -85 grados Celsius. Justo debajo de la mesopausa(capa superior de la mesósfera), el aire es tan frío que incluso el escaso vapor de agua puede ser sublimado en nubes nocturnas visibles al ojo desnudo si la luz solar refleja sobre ellos una hora o dos luego de la puesta de sol. 
\subsection{Estratosfera}
La estratosfera es la segunda capa más baja de la atmósfera de la Tierra. Esta capa se extiende a una altitud de aproximadamente 50 a 55km.
La presión atmosférica en la cima de la estratosfera es de apenas $1/100$ la presión al nivel del mar. La estratosfera contiene la capa de Ozono, la cual es la parte de la atmósfera terrestre que contiene concentraciones relativamente altas de ese gas. 
\subsection{Tropósfera}
La tropósfera es la capa más baja de la atmósfera de la Tierra. Se extiende desde la superficie terrestre a una altura de 12km.
Aunque algunas variaciones ocurren, la temperatura de la tropósfera usualmente decae con la altura ya que la tropósfera se calienta mayormente a través de transferencia de energía desde la Tierra.
Casi todo el vapor de agua y humedad se encuentran en la tropósfera, por lo que es la capa en donde la mayor parte del clima de la Tierra ocurre.
\section{Propiedades Físicas}
\subsection{Presión}
La presión atmosférica promedio al nivel del mar está definida por el Estándar Internacional Atmosférico (ISA) como 101325 pascales. La presión atmósferica es el total de peso del aire sobre unidad de área al punto donde la presión es medida, por lo que ésta varía con la locación y el clima.
\subsection{Temperatura y Velocidad del Sonido}
La temperatura decae con la altitud empezando al nivel del mar, pero las variaciones en este aspecto empiezan por encima de los 11km. Sin embargo en la estratósfera comenzando por encima de los 20km, la temperatura aumenta con la altura debido al calentamiento en la capa de ozono causada por la captura significativa de radiación ultravioleta del Sol.
Dado que en un gas ideal de composición constante la velocidad del sonido depende únicamente de la temperatura y no del gas, presión o densidad, la velocidad del sonido en la atmósfera con la altura toma la forma del complicado perfil de temperatura y no refleja cambios altitudinales en densidad o presión.
\subsection{Densidad y masa}
La densidad del aire a nivel del mar es de aproximadamente $1.2kg/m^{3}$. La densidad no se mide directamente, pero es calculada a partir de mediciones de temperatura, presión y humedad usando la ecuación de estado para aire (una forma de la ley de gas ideal). La densidad atmósferica aumenta conforme la altitud aumenta. Esta variación puede ser aproximadamente modelada usando la fórmula barométrica. 
La masa total de la atmósfera es de $5*1480*10^{18}kg$ con un intervalo anual debido a la vaporización del agua de 1.2 o $1.5*10^{15}kg$, dependiendo si se usan datos de presión sobre la superficie o vapor de agua.
\section{Propiedades Ópticas}
\begin{figure}[h!]
\includegraphics[width=\linewidth]{verde.jpg}
\end{figure}
La radiación solar es energía recibida por la Tierra desde el Sol. La Tierra a su vez también emite radiación de vuelta al espacio pero en longitudes de onda más largas que no podemos ver. Parte de la radiación entrante y saliente es absorbida o reflejada en la atmósfera.
\subsection{Dispersión}
Cuando la luz pasa por la atmósfera terrestre, fotones interactúan con ella a través de la dispersión. Si la luz no interactúa con la atmósfera se le llama radiación directa y es lo que verías al mirar directamente hacia el Sol. Debido a un fenómeno llamado \textit{Dispersión de Rayleigh} longitudes de onda cortas (azules) se dispersan más fácilmente que las longitudes de onda largas (rojas). Es por eso que el cielo se ve azul.
\begin{figure}[h!]
\includegraphics[width=\linewidth]{luna.JPG}
\end{figure}
\subsection{Absorción}
Diferentes moléculas absorben diferentes ondas de radiación.El espectro de absorción combinado de gases en la atmósfera dejan ventanas de baja opacidad, permitiendo la transmisión de ciertas bandas de luz.
\subsection{Emisión}
La emisión es lo opuesto a la absorción y es cuando un objeto emite radiación. Los objetos más calientes tienden a emitir más radiación con longitudes de onda más cortas. Los objetos fríos emiten menos radiación con longitudes de onda más largas. Debido a su temperatura, la atmósfera emite radiación infrarroja. 
\subsection{Índice de Refracción}
El índice de refracción del aire es cercano, pero apenas más grande que 1. Variaciones sistemáticas en el índice de refracción pueden llevar a la curvatura de rayos de luz sobre largos caminos ópticos. Un ejemplo de esto es cuando, bajo ciertas circunstancias, sobre barcos se pueden observar otros buques sobre el horizonte ya que la luz se refracta en la misma dirección de la curvatura de la Tierra.
\section{Circulación}
La circulación atmosférica es el movimiento a gran escala del aire a través de la tropósfera y los términos sobre los cuales se distribuye el calor en la Tierra. La estructura a gran escala de la circulación atmosférica varía año con año, pero la estructura básica se mantiene relativamente constante debido a que se determina a partir de la velocidad de rotación de la Tierra y la diferencia en radiación solar entre ecuador y polos.
\section{Bibliografía}
1. NASA (2018, Enero 29).Gateway to Astronaut Photos of Earth. Recuperado de 
2. Zimmer, C (2013, Octubre 3). Earth's Oxygen: A Mystery Easy to Take for Granted. Recuperado de http://www.nytimes.com/2013/10/03/science/earths-oxygen-a-mystery-easy-to-take-for-granted.html
3. St. Fleur,N (2017,Mayo 19). Spotting Mysterious Twinkles on Earth From a Million Miles Away. Recuperado de https://www.nytimes.com/2017/05/19/science/dscovr-satellite-ice-glints-earth-atmosphere.html
\section{Apéndice}
1.¿Qué fue lo que más te llamó la atención de esta actividad?
El hecho de poder investigar acerca de la atmósfera terrestre y a su vez el aprender a utilizar LaTex.
2.¿Qué fue lo que se te hizo menos interesante?
Todo me pareció interesante.
3.¿Qué cambios harías para mejorar esta actividad? 
Nada.
4.¿Cuál es tu primera impresión de uso de LATEX?
Es fácil acostumbrarse a él y parece muy útil.
5.¿El tiempo sugerido para esta actividad fue suficiente? 
Sí.
6.¿Encontraste algún documento o recurso en línea útil que quisieras compartir con los demás? 
LaTex Wikibook.

\end{document}
